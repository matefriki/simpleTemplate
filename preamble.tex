
% FCCordoba Template version                            31 / 08 / 2017
 
 
 \usepackage[utf8]{inputenc}	% Accents and other symbols
 \usepackage{bbm}	% Permet fer \1
 %%% Nota: Si la plantilla no compila, comenta els dos packages anteriors. \usepackage{amsfonts,amssymb}                          % Símbols matemàtics
 \usepackage{mathrsfs}                                  % Lletra cal·ligrafica (millor que \mathcal)
 \usepackage{amsmath, amsthm, amsfonts, amssymb}        % Mode matemàtic, TheoremStyle, Simbols matematics
%\usepackage{bookman}                                   % BarreroTimes
 \usepackage{verbatim}                                  % Verbatim + comentaris multilínia
 \usepackage{graphicx,subcaption}                       % Fotos i subfotos
 \usepackage[free-standing-units]{siunitx}              % Sistema internacional
 \usepackage[siunitx, american]{circuitikz}             % Circuits
 \usepackage{float}                                     % Posa les figures on toca amb [H]
 \usepackage{anysize}                                   % Per usar \marginsize{}{}{}{}
 \marginsize{2cm}{2cm}{1cm}{4cm}     % {L}{R}{U}{D}
 \usepackage{tikz-cd}                                   % Diagrames commutatius
% \usetikzlibrary{babel}                                % Evita interferència entre tikz-cd i babel
\usepackage{helvet} \renewcommand{\familydefault}{\sfdefault}        % Estas dos lineas               % ponen Arial
\usepackage[toc,page]{appendix}                         % for appendices
\usepackage{enumerate}					% enumerate options
\usepackage{hyperref}					%for hyperrefferences  (needed in bibliography
\usepackage[square,comma,numbers,sort&compress]{natbib} % bibliography stuff (see http://www.colorado.edu/physics/phys4610/phys4610_sp12/bibtex_guide.pdf)

\theoremstyle{definition} \newtheorem{definition}{Definition}%[chapter]
\theoremstyle{definition} \newtheorem{axiom}{Axiom}%[chapter]
\theoremstyle{plain} \newtheorem{theorem}{Theorem}%[chapter]
\theoremstyle{plain} \newtheorem{proposition}{Proposition}%[chapter]
\theoremstyle{plain} \newtheorem*{lemma}{Lemma}
\theoremstyle{remark} \newtheorem*{corollary}{Corollary}
\theoremstyle{remark} \newtheorem*{remark}{Remark}
\theoremstyle{remark} \newtheorem*{notation}{Notation}
%\theoremstyle{remark} \newtheorem{Proof}{Proof}

\renewenvironment{proof}{\\ \textit{Proof.  } \begin{rm} \small }{ \normalsize $\qed$ \end{rm}}


%Conjunts importants
\def\NN{\mathbb N}   % Naturals
\def\ZZ{\mathbb Z}   % Enters
\def\QQ{\mathbb Q}   % Racionals
\def\RR{\mathbb R}   % Reals
\def\CC{\mathbb C}   % Complexos
\def\HH{\mathbb H}   % Quaternions, semiplà superior complex
\def\AA{\mathbb A}   % Espai afí en general
\def\EE{\mathbb E}   % Extensió d'un cos, esperança
\def\FF{\mathbb F}   % Cos primer, cos finit
\def\KK{\mathbb K}   % Cos en general
\def\PP{\mathbb P}   % Espai projectiu, nombres primers
\def\DD{\mathbb D}   % Disc unitat complex
\def\SS{\mathbb S}   % Esfera
\def\TT{\mathbb T}   % Tor (n-dimensional)
\def\XX{\mathbb X}

% Per escriure menys
\def\oo{\infty}                                         % Infinit
\def\P{\mathscr P}                                      % Conjunt potència
\def\A{\mathscr A}                                      % Àlgebra, sigma-àlgebra
\def\B{\mathscr B}                                      % Base
\def\T{\mathscr T}                                      % Topologia
\def\L{\mathscr L}                                      % Espai d'homomorfismes, derivada de Lie
\def\C{\mathcal C}                                      % Funcions contínues, derivables amb continuïtat
\def\O{\mathcal O}										% big O notation
\def\D{\mathscr D}                                      % Funcions derivables
\def\mm{\mathfrak M}                                    % Matrius
%\def\ss{\mathfrak S}                                    % Grup simètric
\def\aa{\mathfrak A}                                    % Grup alternat
\def\Re#1{\mathfrak {Re}\left\{#1\right\}}              % Part real
\def\Im#1{\mathfrak {Im}\left\{#1\right\}}              % Part imaginària
\def\1{\mathbbm{1}}                                     % Funció indicadora
\def\<{\langle}                                         % subespai/ideal/subgrup -
\def\>{\rangle}%                                          generat per, producte escalar
\def\sect{\mathsection}                                 % Secció
\def\pgph{\mathparagraph}                               % Fi de paràgraf
\def\qed{\hfill\square}                                 % Quadrat blanc de Q.E.D.

%Calia
\def\phi{\varphi}       %%% Aquesta comanda inhabilita el "\phi" lleig
\def\eps{\varepsilon}   %\epsilon


\DeclareMathOperator*{\argmin}{arg\,min}                                 % Punt on s'assoleix el mínim
\DeclareMathOperator*{\argmax}{arg\,max}                                 % Punt on s'assoleix el màxim
\DeclareMathOperator*{\mex}{mex}                                         % Mínim ordinal exclòs
\DeclareMathOperator*{\sgn}{sgn}                                         % Signe
\DeclareMathOperator*{\im}{Im}                                           % Imatge
\DeclareMathOperator*{\Tr}{Tr}                                           % Traça
\DeclareMathOperator*{\Id}{Id}                                           % Identitat
\DeclareMathOperator*{\supp}{supp}                                       % Suport
\DeclareMathOperator*{\prolim}{\underleftarrow{\rm{proj\,lim}}}          % Límit projectiu
\DeclareMathOperator*{\indlim}{\underrightarrow{\rm{ind\,lim}}}          % Límit inductiu

%Trigonometria bàsica
\DeclareMathOperator*{\tg}{tg}                          % tg(·)          = sin(·)/cos(·)
\DeclareMathOperator*{\cosec}{cosec}                    % cosec(·)       = 1/sin(·)
\DeclareMathOperator*{\cotg}{cotg}                      % cotg(·)        = cos(·)/sin(·)

\DeclareMathOperator*{\arctg}{arctg}
\DeclareMathOperator*{\arcsec}{arcsec}
\DeclareMathOperator*{\arccosec}{arccosec}
\DeclareMathOperator*{\arccotg}{arccotg}

\DeclareMathOperator*{\tgh}{tgh}                        % tgh(·)         = sinh(·)/cosh(·)
\DeclareMathOperator*{\sech}{sech}                      % sech(·)        = 1/cosh(·)
\DeclareMathOperator*{\cosech}{cosech}                  % cosech(·)      = 1/sinh(·)
\DeclareMathOperator*{\cotgh}{cotgh}                    % cotgh(·)       = cosh(·)/sinh(·)

\DeclareMathOperator*{\arcsinh}{arcsinh}
\DeclareMathOperator*{\arccosh}{arccosh}
\DeclareMathOperator*{\arctgh}{arctgh}
\DeclareMathOperator*{\arcsech}{arcsech}
\DeclareMathOperator*{\arccosech}{arccosech}
\DeclareMathOperator*{\arccotgh}{arccotgh}

% Operadors grans
\DeclareMathOperator*{\bigcomma}{\raisebox{0.9ex}{\Huge ,}}  % Comatori                    %% e.g. $\RR = \left\{\bigcomma\limits_{x\in\RR} x \right\}$
\DeclareMathOperator*{\bigtimes}{\text{\Large $\times$}}     % Cartesionatori              %% e.g. $\bigtimes\limits_{i\in B} A = A^B :=\{f:B\to A\}$
\DeclareMathOperator*{\bigvoid}{\text{\Large $\O$}}          % Concatenatori               %% e.g. $\bigvoid\limits_{i=1}^n \left(\sum\limits_{j_i=1}^n\right) 1 = n^n$
\DeclareMathOperator*{\bigequals}{\text{\Large $=$}}         % Igualatori                  %% e.g. $\bigequals\limits_{n=1}^\oo \sum\limits_{i=1}^n \dfrac1n$
\DeclareMathOperator*{\bigle}{\text{\Large $\le$}}           % Creixentatori               %% e.g. $0<\bigle\limits_{n=1}^\oo a_n<k\implies\exists\lim\limits_{n\to\oo}a_n\leq k$
\DeclareMathOperator*{\bigge}{\text{\Large $\ge$}}           % Decreixentatori
\DeclareMathOperator*{\bigless}{\text{\Large $<$}}           % Creixentatori estricte
\DeclareMathOperator*{\biggreater}{\text{\Large $>$}}        % Decreixentatori estricte
\DeclareMathOperator*{\bigsse}{\text{\Large $\sse$}}         % Inclusionatori              %% e.g. $\O\sse\bigsse\limits_{n=1}^\oo (-n,n)\sse\R$
\DeclareMathOperator*{\bigspse}{\text{\Large $\spse$}}       % Antiinclusionatori
\DeclareMathOperator*{\bigsss}{\text{\Large $\sss$}}         % Inclusionatori estricte
\DeclareMathOperator*{\bigssne}{\text{\Large $\ssne$}}       % Inclusionatori estricte
\DeclareMathOperator*{\bigssps}{\text{\Large $\ssps$}}       % Antiinclusionatori estricte
\DeclareMathOperator*{\bigspsne}{\text{\Large $\spsne$}}     % Antiinclusionatori estricte
\DeclareMathOperator*{\bigni}{\text{\Large $\ni$}}           % Contenatori                 %% e.g. $\nexists\bigni\limits_{i=0}^\oo A_i$
\DeclareMathOperator*{\bigin}{\text{\Large $\in$}}           % Pertanyatori
\DeclareMathOperator*{\bigo}{\bigcirc}                       % Compositori                 %% e.g. $\bigequals\limits_{x\in\R}\left(\bigo\limits_{i=1}^\oo \cos\right) (x)$
\DeclareMathOperator*{\bigfrac}{\raisebox{-.5ex}{\Large K}}  % K-atori                     %% e.g. $\bigfrac\limits_{i=1}^\oo(b_i:c_i):=\cfrac{b_1}{c_1+\cfrac{b_2}{c_2+\ddots}}$
            %%% Nota: El sumadirectori (\bigoplus), els dos uniodisjuntatoris (\bigsqcup, \biguplus) i el tensionatori (\bigotimes) ja estan implementats per defecte.

% Permet fer peus de pàgina amb altres símbols
\makeatletter
\def\@xfootnote[#1]{\protected@xdef\@thefnmark{#1}\@footnotemark\@footnotetext}
\makeatother

% Permet fer arrels amb palet
\newcommand{\sqrtt}[1][\hphantom{3}]{%
  \def\DHLindex{#1}\mathpalette\DHLhksqrt}
\def\DHLhksqrt#1#2{%
  \setbox0=\hbox{$#1\sqrt[\DHLindex]{#2\,}$}\dimen0=\ht0
  \advance\dimen0-0.2\ht0
  \setbox2=\hbox{\vrule height\ht0 depth -\dimen0}%
  {\box0\lower0.4pt\box2}}

% Objectes no textuals
\begin{comment}
--------------------Taula--------------------

    http://www.tablesgenerator.com

-------------Diagrama Commutatiu-------------

    http://osl.ugr.es/CTAN/graphics/pgf/contrib/tikz-cd/tikz-cd-doc.pdf

-------------------Circuit-------------------

    http://texdoc.net/texmf-dist/doc/latex/circuitikz/circuitikzmanual.pdf

--------------------Foto---------------------

    \begin{figure}[H]
        \centering
        \includegraphics[width=\linewidth]{Imatges/Loremipsum.png}
    \end{figure}

-------------------Subfoto-------------------

    \begin{figure}[H]
        \begin{subfigure}[b]{.5\linewidth}
            \centering
            \includegraphics[width=\linewidth]{Imatges/Lorem.png}
        \end{subfigure}
        \begin{subfigure}[b]{.5\linewidth}
            \centering
            \includegraphics[width=\linewidth]{Imatges/Ipsum.png}
        \end{subfigure}
    \end{figure}

\end{comment}

% Per veure si falta res en un treball (el típic "ho deixo per després")
\DeclareMathOperator*{\completar}{\_\_\_\_\_\_\_\_\_\_}
